\documentclass[]{tufte-handout}

% ams
\usepackage{amssymb,amsmath}

\usepackage{ifxetex,ifluatex}
\usepackage{fixltx2e} % provides \textsubscript
\ifnum 0\ifxetex 1\fi\ifluatex 1\fi=0 % if pdftex
  \usepackage[T1]{fontenc}
  \usepackage[utf8]{inputenc}
\else % if luatex or xelatex
  \makeatletter
  \@ifpackageloaded{fontspec}{}{\usepackage{fontspec}}
  \makeatother
  \defaultfontfeatures{Ligatures=TeX,Scale=MatchLowercase}
  \makeatletter
  \@ifpackageloaded{soul}{
     \renewcommand\allcapsspacing[1]{{\addfontfeature{LetterSpace=15}#1}}
     \renewcommand\smallcapsspacing[1]{{\addfontfeature{LetterSpace=10}#1}}
   }{}
  \makeatother

\fi

% graphix
\usepackage{graphicx}
\setkeys{Gin}{width=\linewidth,totalheight=\textheight,keepaspectratio}

% booktabs
\usepackage{booktabs}

% url
\usepackage{url}

% hyperref
\usepackage{hyperref}

% units.
\usepackage{units}


\setcounter{secnumdepth}{-1}

% citations


% pandoc syntax highlighting
\usepackage{color}
\usepackage{fancyvrb}
\newcommand{\VerbBar}{|}
\newcommand{\VERB}{\Verb[commandchars=\\\{\}]}
\DefineVerbatimEnvironment{Highlighting}{Verbatim}{commandchars=\\\{\}}
% Add ',fontsize=\small' for more characters per line
\newenvironment{Shaded}{}{}
\newcommand{\AlertTok}[1]{\textcolor[rgb]{1.00,0.00,0.00}{\textbf{#1}}}
\newcommand{\AnnotationTok}[1]{\textcolor[rgb]{0.38,0.63,0.69}{\textbf{\textit{#1}}}}
\newcommand{\AttributeTok}[1]{\textcolor[rgb]{0.49,0.56,0.16}{#1}}
\newcommand{\BaseNTok}[1]{\textcolor[rgb]{0.25,0.63,0.44}{#1}}
\newcommand{\BuiltInTok}[1]{#1}
\newcommand{\CharTok}[1]{\textcolor[rgb]{0.25,0.44,0.63}{#1}}
\newcommand{\CommentTok}[1]{\textcolor[rgb]{0.38,0.63,0.69}{\textit{#1}}}
\newcommand{\CommentVarTok}[1]{\textcolor[rgb]{0.38,0.63,0.69}{\textbf{\textit{#1}}}}
\newcommand{\ConstantTok}[1]{\textcolor[rgb]{0.53,0.00,0.00}{#1}}
\newcommand{\ControlFlowTok}[1]{\textcolor[rgb]{0.00,0.44,0.13}{\textbf{#1}}}
\newcommand{\DataTypeTok}[1]{\textcolor[rgb]{0.56,0.13,0.00}{#1}}
\newcommand{\DecValTok}[1]{\textcolor[rgb]{0.25,0.63,0.44}{#1}}
\newcommand{\DocumentationTok}[1]{\textcolor[rgb]{0.73,0.13,0.13}{\textit{#1}}}
\newcommand{\ErrorTok}[1]{\textcolor[rgb]{1.00,0.00,0.00}{\textbf{#1}}}
\newcommand{\ExtensionTok}[1]{#1}
\newcommand{\FloatTok}[1]{\textcolor[rgb]{0.25,0.63,0.44}{#1}}
\newcommand{\FunctionTok}[1]{\textcolor[rgb]{0.02,0.16,0.49}{#1}}
\newcommand{\ImportTok}[1]{#1}
\newcommand{\InformationTok}[1]{\textcolor[rgb]{0.38,0.63,0.69}{\textbf{\textit{#1}}}}
\newcommand{\KeywordTok}[1]{\textcolor[rgb]{0.00,0.44,0.13}{\textbf{#1}}}
\newcommand{\NormalTok}[1]{#1}
\newcommand{\OperatorTok}[1]{\textcolor[rgb]{0.40,0.40,0.40}{#1}}
\newcommand{\OtherTok}[1]{\textcolor[rgb]{0.00,0.44,0.13}{#1}}
\newcommand{\PreprocessorTok}[1]{\textcolor[rgb]{0.74,0.48,0.00}{#1}}
\newcommand{\RegionMarkerTok}[1]{#1}
\newcommand{\SpecialCharTok}[1]{\textcolor[rgb]{0.25,0.44,0.63}{#1}}
\newcommand{\SpecialStringTok}[1]{\textcolor[rgb]{0.73,0.40,0.53}{#1}}
\newcommand{\StringTok}[1]{\textcolor[rgb]{0.25,0.44,0.63}{#1}}
\newcommand{\VariableTok}[1]{\textcolor[rgb]{0.10,0.09,0.49}{#1}}
\newcommand{\VerbatimStringTok}[1]{\textcolor[rgb]{0.25,0.44,0.63}{#1}}
\newcommand{\WarningTok}[1]{\textcolor[rgb]{0.38,0.63,0.69}{\textbf{\textit{#1}}}}

% longtable

% multiplecol
\usepackage{multicol}

% strikeout
\usepackage[normalem]{ulem}

% morefloats
\usepackage{morefloats}


% tightlist macro required by pandoc >= 1.14
\providecommand{\tightlist}{%
  \setlength{\itemsep}{0pt}\setlength{\parskip}{0pt}}

% title / author / date
\title{ANTsR TUT}
\author{Leonardo Cerliani}
\date{09 January, 2021}


\begin{document}

\maketitle




\hypertarget{before-you-start}{%
\section{Before you start}\label{before-you-start}}

This tutorial for ANTsR is based on the Rmd
\href{https://github.com/stnava/ANTsTutorial/blob/master/registration/antsRegistrationIntro.Rmd}{notebook
by Brian Avants on antsRegistration in R}.

The Rmd version was evaluated riding on the Storm, but it should work
everywhere.

\hypertarget{read-images}{%
\section{Read images}\label{read-images}}

\begin{Shaded}
\begin{Highlighting}[]
\KeywordTok{library}\NormalTok{(ANTsR)}
\KeywordTok{library}\NormalTok{(ggplot2)}

\CommentTok{# Read images}
\NormalTok{r16 =}\StringTok{ }\KeywordTok{antsImageRead}\NormalTok{( }\KeywordTok{getANTsRData}\NormalTok{( }\StringTok{"r16"}\NormalTok{ ) )}
\NormalTok{r64 =}\StringTok{ }\KeywordTok{antsImageRead}\NormalTok{( }\KeywordTok{getANTsRData}\NormalTok{( }\StringTok{"r64"}\NormalTok{ ) )}
\end{Highlighting}
\end{Shaded}

\hypertarget{visualization}{%
\section{Visualization}\label{visualization}}

\hypertarget{define-functions-for-plotting}{%
\subsection{Define functions for
plotting}\label{define-functions-for-plotting}}

\begin{Shaded}
\begin{Highlighting}[]
\CommentTok{# Only one image}
\NormalTok{antsplot.single <-}\StringTok{ }\ControlFlowTok{function}\NormalTok{(image)\{}
  \KeywordTok{invisible}\NormalTok{(}
    \KeywordTok{plot}\NormalTok{(}
\NormalTok{      image,}
      \DataTypeTok{doCropping =}\NormalTok{ F}
\NormalTok{    )  }
\NormalTok{  )}
\NormalTok{\}}


\CommentTok{# Simple overlay with alpha}
\NormalTok{antsplot.olay <-}\StringTok{ }\ControlFlowTok{function}\NormalTok{(bg, olay, }\DataTypeTok{alpha=}\FloatTok{0.5}\NormalTok{)\{}
  \KeywordTok{invisible}\NormalTok{(}
    \KeywordTok{plot}\NormalTok{(}
\NormalTok{      bg, olay,}
      \DataTypeTok{color.overlay=}\StringTok{'red'}\NormalTok{,}
      \DataTypeTok{doCropping =}\NormalTok{ F,}
      \DataTypeTok{alpha=}\NormalTok{alpha,}
      \DataTypeTok{window.overlay =} \KeywordTok{quantile}\NormalTok{(olay, }\KeywordTok{c}\NormalTok{(}\FloatTok{0.72}\NormalTok{,}\DecValTok{1}\NormalTok{))}
\NormalTok{    )  }
\NormalTok{  )}
\NormalTok{\}}


\CommentTok{# Overlay canny borders}
\NormalTok{antsplot.canny <-}\StringTok{ }\ControlFlowTok{function}\NormalTok{(bg, olay, }\DataTypeTok{alpha=}\DecValTok{1}\NormalTok{)\{}
\NormalTok{  canned_olay =}\StringTok{ }\KeywordTok{iMath}\NormalTok{(olay, }\StringTok{"Canny"}\NormalTok{, }\DecValTok{3}\NormalTok{,}\DecValTok{3}\NormalTok{,}\DecValTok{3}\NormalTok{)}
  \KeywordTok{invisible}\NormalTok{(}
    \KeywordTok{plot}\NormalTok{(}
\NormalTok{      bg, canned_olay,}
      \DataTypeTok{color.overlay=}\StringTok{'red'}\NormalTok{,}
      \DataTypeTok{doCropping =}\NormalTok{ F,}
      \DataTypeTok{alpha=}\NormalTok{alpha}
\NormalTok{    )  }
\NormalTok{  )}
\NormalTok{\}}
\end{Highlighting}
\end{Shaded}

\hypertarget{plot-single-image}{%
\subsection{Plot single image}\label{plot-single-image}}

\begin{Shaded}
\begin{Highlighting}[]
\KeywordTok{antsplot.single}\NormalTok{(r16)}
\end{Highlighting}
\end{Shaded}

\begin{verbatim}
## Warning in fun(libname, pkgname): couldn't connect to display ":0"
\end{verbatim}

\includegraphics{ANTsR_TUT_files/figure-latex/unnamed-chunk-3-1}

\hypertarget{plot-overlay-with-transparency}{%
\subsection{Plot overlay with
transparency}\label{plot-overlay-with-transparency}}

\begin{Shaded}
\begin{Highlighting}[]
\KeywordTok{antsplot.olay}\NormalTok{(r16,r64)}
\end{Highlighting}
\end{Shaded}

\includegraphics{ANTsR_TUT_files/figure-latex/unnamed-chunk-4-1}

\hypertarget{plot-borders-of-overlay}{%
\subsection{Plot borders of overlay}\label{plot-borders-of-overlay}}

\begin{Shaded}
\begin{Highlighting}[]
\KeywordTok{antsplot.canny}\NormalTok{(r16,r64)}
\end{Highlighting}
\end{Shaded}

\includegraphics{ANTsR_TUT_files/figure-latex/unnamed-chunk-5-1}

\hypertarget{estimate-registration}{%
\section{Estimate Registration}\label{estimate-registration}}

See \texttt{?antsRegistration()} for different typesofTransform

\hypertarget{translation}{%
\subsection{Translation}\label{translation}}

\begin{Shaded}
\begin{Highlighting}[]
\NormalTok{trans_reg =}\StringTok{ }\KeywordTok{antsRegistration}\NormalTok{(}
  \DataTypeTok{fixed=}\NormalTok{r16, }
  \DataTypeTok{moving=}\NormalTok{r64, }
  \DataTypeTok{typeofTransform =} \StringTok{'Translation'}
\NormalTok{)}

\KeywordTok{readAntsrTransform}\NormalTok{(trans_reg}\OperatorTok{$}\NormalTok{fwdtransforms)}
\end{Highlighting}
\end{Shaded}

\begin{verbatim}
## antsrTransform
##   Dimensions    : 2 
##   Precision     : float 
##   Type          : AffineTransform
\end{verbatim}

\begin{Shaded}
\begin{Highlighting}[]
\KeywordTok{antsplot.canny}\NormalTok{(r16, trans_reg}\OperatorTok{$}\NormalTok{warpedmovout)}
\end{Highlighting}
\end{Shaded}

\includegraphics{ANTsR_TUT_files/figure-latex/unnamed-chunk-6-1}

\hypertarget{rigid-registration}{%
\subsection{Rigid registration}\label{rigid-registration}}

\begin{Shaded}
\begin{Highlighting}[]
\NormalTok{rigid_reg =}\StringTok{ }\KeywordTok{antsRegistration}\NormalTok{(}
  \DataTypeTok{fixed=}\NormalTok{r16, }
  \DataTypeTok{moving=}\NormalTok{r64, }
  \DataTypeTok{typeofTransform =} \StringTok{'Rigid'}
\NormalTok{)}


\KeywordTok{antsplot.canny}\NormalTok{(r16, rigid_reg}\OperatorTok{$}\NormalTok{warpedmovout)}
\end{Highlighting}
\end{Shaded}

\includegraphics{ANTsR_TUT_files/figure-latex/unnamed-chunk-7-1}

\hypertarget{affine-registration}{%
\subsection{Affine registration}\label{affine-registration}}

\begin{Shaded}
\begin{Highlighting}[]
\NormalTok{affine_reg =}\StringTok{ }\KeywordTok{antsRegistration}\NormalTok{(}
  \DataTypeTok{fixed=}\NormalTok{r16, }
  \DataTypeTok{moving=}\NormalTok{r64, }
  \DataTypeTok{typeofTransform =} \StringTok{'Affine'}\NormalTok{, }
  \DataTypeTok{verbose =}\NormalTok{ F}
\NormalTok{)}

\KeywordTok{antsplot.canny}\NormalTok{(r16, affine_reg}\OperatorTok{$}\NormalTok{warpedmovout)}
\end{Highlighting}
\end{Shaded}

\includegraphics{ANTsR_TUT_files/figure-latex/unnamed-chunk-8-1}

\hypertarget{syn-registration}{%
\subsection{SyN registration}\label{syn-registration}}

\begin{Shaded}
\begin{Highlighting}[]
\NormalTok{syn_reg =}\StringTok{ }\KeywordTok{antsRegistration}\NormalTok{(}
  \DataTypeTok{fixed =}\NormalTok{ r16,}
  \DataTypeTok{moving =}\NormalTok{ r64,}
  \DataTypeTok{typeofTransform =} \StringTok{'SyN'}
\NormalTok{)}

\KeywordTok{antsplot.canny}\NormalTok{(r16, syn_reg}\OperatorTok{$}\NormalTok{warpedmovout)}
\end{Highlighting}
\end{Shaded}

\includegraphics{ANTsR_TUT_files/figure-latex/unnamed-chunk-9-1}

\hypertarget{syncc-registration}{%
\subsection{SyNCC registration}\label{syncc-registration}}

\begin{Shaded}
\begin{Highlighting}[]
\NormalTok{synCC_reg =}\StringTok{ }\KeywordTok{antsRegistration}\NormalTok{(}
  \DataTypeTok{fixed =}\NormalTok{ r16,}
  \DataTypeTok{moving =}\NormalTok{ r64,}
  \DataTypeTok{typeofTransform =} \StringTok{'SyNCC'}
\NormalTok{)}

\KeywordTok{antsplot.canny}\NormalTok{(r16, synCC_reg}\OperatorTok{$}\NormalTok{warpedmovout)}
\end{Highlighting}
\end{Shaded}

\includegraphics{ANTsR_TUT_files/figure-latex/unnamed-chunk-10-1}

\hypertarget{mutual-information-metric}{%
\section{Mutual Information metric}\label{mutual-information-metric}}

To estimate the goodness of different kinds of registration we can plot
the MI metric against the number of parameters

\begin{Shaded}
\begin{Highlighting}[]
\NormalTok{MI <-}\StringTok{ }\KeywordTok{c}\NormalTok{(}
  \KeywordTok{antsImageMutualInformation}\NormalTok{( r16, r64), }
  \KeywordTok{antsImageMutualInformation}\NormalTok{( r16, trans_reg}\OperatorTok{$}\NormalTok{warpedmovout), }
  \KeywordTok{antsImageMutualInformation}\NormalTok{( r16, rigid_reg}\OperatorTok{$}\NormalTok{warpedmovout),}
  \KeywordTok{antsImageMutualInformation}\NormalTok{( r16, affine_reg}\OperatorTok{$}\NormalTok{warpedmovout),}
  \KeywordTok{antsImageMutualInformation}\NormalTok{( r16, syn_reg}\OperatorTok{$}\NormalTok{warpedmovout),}
  \KeywordTok{antsImageMutualInformation}\NormalTok{( r16, synCC_reg}\OperatorTok{$}\NormalTok{warpedmovout)}
\NormalTok{)}

\NormalTok{nparameters =}\StringTok{ }\KeywordTok{c}\NormalTok{( }\DecValTok{0}\NormalTok{, }\DecValTok{2}\NormalTok{, }\DecValTok{4}\NormalTok{, }\DecValTok{8}\NormalTok{, }\DecValTok{16}\NormalTok{, }\DecValTok{32}\NormalTok{ )}
\KeywordTok{plot}\NormalTok{( nparameters, MI, }\DataTypeTok{type=}\StringTok{'l'}\NormalTok{, }\DataTypeTok{main=}\StringTok{'Similarity vs number of parameters'}\NormalTok{ )}
\end{Highlighting}
\end{Shaded}

\includegraphics{ANTsR_TUT_files/figure-latex/unnamed-chunk-11-1}

\hypertarget{jacobian-determinant-image}{%
\section{Jacobian determinant image}\label{jacobian-determinant-image}}

From Brian Avants' notebook on antsRegistration:

``What does ``differentiable map with differentiable inverse'' mean? The
diffeomorphism is like a road between the two images - we can go back
and forth along it.

It also means that if we compose the mapping from A to B with the
mapping from B to A, we get the identity.

We can check this by looking at the result of the composition of SyN's
forward and inverse maps."

\hypertarget{forward-warpfield}{%
\subsection{Forward warpfield}\label{forward-warpfield}}

\begin{Shaded}
\begin{Highlighting}[]
\NormalTok{grid_fw =}\StringTok{ }\KeywordTok{createWarpedGrid}\NormalTok{( r16, }
                            \DataTypeTok{fixedReferenceImage =}\NormalTok{ r16, }
                            \DataTypeTok{transform =}\NormalTok{ syn_reg}\OperatorTok{$}\NormalTok{fwdtransforms[}\DecValTok{1}\NormalTok{] }
\NormalTok{                          )}

\KeywordTok{invisible}\NormalTok{(}\KeywordTok{plot}\NormalTok{( grid_fw ))}
\end{Highlighting}
\end{Shaded}

\includegraphics{ANTsR_TUT_files/figure-latex/unnamed-chunk-12-1}

\hypertarget{inverse-warpfield}{%
\subsection{Inverse warpfield}\label{inverse-warpfield}}

\begin{Shaded}
\begin{Highlighting}[]
\NormalTok{grid_backwd =}\StringTok{ }\KeywordTok{createWarpedGrid}\NormalTok{( r16, }
                                 \DataTypeTok{fixedReferenceImage =}\NormalTok{ r16, }
                                 \DataTypeTok{transform =}\NormalTok{ syn_reg}\OperatorTok{$}\NormalTok{invtransforms[}\DecValTok{2}\NormalTok{] }
\NormalTok{                              )}
\KeywordTok{invisible}\NormalTok{(}\KeywordTok{plot}\NormalTok{( grid_backwd ))}
\end{Highlighting}
\end{Shaded}

\includegraphics{ANTsR_TUT_files/figure-latex/unnamed-chunk-13-1}

\hypertarget{composition-of-forward-and-inverse}{%
\subsection{Composition of forward and
inverse}\label{composition-of-forward-and-inverse}}

\begin{Shaded}
\begin{Highlighting}[]
\NormalTok{emptygrid =}\StringTok{ }\KeywordTok{createWarpedGrid}\NormalTok{( r16, }\DataTypeTok{fixedReferenceImage =}\NormalTok{ r16 )}

\NormalTok{invidmap =}\StringTok{ }\KeywordTok{antsApplyTransforms}\NormalTok{( r16, emptygrid, }
   \KeywordTok{c}\NormalTok{( syn_reg}\OperatorTok{$}\NormalTok{invtransforms, syn_reg}\OperatorTok{$}\NormalTok{fwdtransforms ), }
   \DataTypeTok{whichtoinvert =} \KeywordTok{c}\NormalTok{( T,F,F,F ))  }\CommentTok{# tricky stuff here}

\KeywordTok{invisible}\NormalTok{(}\KeywordTok{plot}\NormalTok{( invidmap ))}
\end{Highlighting}
\end{Shaded}

\includegraphics{ANTsR_TUT_files/figure-latex/unnamed-chunk-14-1}

``This ``inverse identity'' constraint is built into SyN such that we
enforce consistency in the digital domain.

What this means is - any ``shape change'' is encoded losslessly into the
deformation field.

\textbf{Non-diffeomorphic} maps may lose information or may be
completely uninterpretable, statistically.

\textbf{One way to check this is to investigate the jacobian. They
should be positive which indicates that the topology of the image space
is preserved ( no folding and no holes or tears are created ).}"

\begin{Shaded}
\begin{Highlighting}[]
\NormalTok{syn_reg_jac =}\StringTok{ }\KeywordTok{createJacobianDeterminantImage}\NormalTok{( r16, }
\NormalTok{                                              syn_reg}\OperatorTok{$}\NormalTok{fwdtransforms[}\DecValTok{1}\NormalTok{], }
                                              \DataTypeTok{geom =} \OtherTok{TRUE}\NormalTok{,}
                                              \DataTypeTok{doLog =}\NormalTok{ T)}

\NormalTok{synCC_reg_jac =}\StringTok{ }\KeywordTok{createJacobianDeterminantImage}\NormalTok{( r16, }
\NormalTok{                                                synCC_reg}\OperatorTok{$}\NormalTok{fwdtransforms[}\DecValTok{1}\NormalTok{], }
                                                \DataTypeTok{geom =} \OtherTok{TRUE}\NormalTok{,}
                                                \DataTypeTok{doLog =}\NormalTok{ T)}
\end{Highlighting}
\end{Shaded}

\hypertarget{is-there-a-difference-between-the-two-jacobians-paired-t-test.}{%
\section{Is there a difference between the two jacobians? Paired
t-test.}\label{is-there-a-difference-between-the-two-jacobians-paired-t-test.}}

\begin{Shaded}
\begin{Highlighting}[]
\NormalTok{mask =}\StringTok{ }\KeywordTok{getMask}\NormalTok{( r16 ) }\OperatorTok\StringTok{ }\KeywordTok{morphology}\NormalTok{(}\StringTok{"dilate"}\NormalTok{,}\DecValTok{3}\NormalTok{)}
\KeywordTok{print}\NormalTok{( }\KeywordTok{t.test}\NormalTok{( syn_reg_jac[ mask }\OperatorTok{==}\StringTok{ }\DecValTok{1}\NormalTok{ ], synCC_reg_jac[ mask }\OperatorTok{==}\StringTok{ }\DecValTok{1}\NormalTok{], }\DataTypeTok{paired=}\OtherTok{TRUE}\NormalTok{  ) )}
\end{Highlighting}
\end{Shaded}

\begin{verbatim}
## 
##  Paired t-test
## 
## data:  syn_reg_jac[mask == 1] and synCC_reg_jac[mask == 1]
## t = 25.882, df = 19684, p-value < 2.2e-16
## alternative hypothesis: true difference in means is not equal to 0
## 95 percent confidence interval:
##  0.06089004 0.07086822
## sample estimates:
## mean of the differences 
##              0.06587913
\end{verbatim}

\begin{Shaded}
\begin{Highlighting}[]
\KeywordTok{library}\NormalTok{( ggplot2 )}

\NormalTok{n =}\StringTok{ }\KeywordTok{sum}\NormalTok{( mask }\OperatorTok{==}\StringTok{ }\DecValTok{1}\NormalTok{)}

\CommentTok{# SyN uses MI, SyNCC uses CC}
\NormalTok{mydf =}\StringTok{ }\KeywordTok{data.frame}\NormalTok{( }
  \DataTypeTok{registration =} \KeywordTok{c}\NormalTok{( }\KeywordTok{rep}\NormalTok{( }\StringTok{"cc"}\NormalTok{, n ) , }\KeywordTok{rep}\NormalTok{(}\StringTok{"mi"}\NormalTok{, n ) ),}
  \DataTypeTok{jacobian =} \KeywordTok{c}\NormalTok{( synCC_reg_jac[ mask }\OperatorTok{==}\StringTok{ }\DecValTok{1}\NormalTok{ ], syn_reg_jac[ mask }\OperatorTok{==}\StringTok{ }\DecValTok{1}\NormalTok{] ) )}

\KeywordTok{ggplot}\NormalTok{( mydf, }\KeywordTok{aes}\NormalTok{(jacobian, }\DataTypeTok{fill=}\NormalTok{registration )) }\OperatorTok{+}\StringTok{ }\KeywordTok{geom_density}\NormalTok{(}\DataTypeTok{alpha =} \FloatTok{0.2}\NormalTok{)}
\end{Highlighting}
\end{Shaded}

\includegraphics{ANTsR_TUT_files/figure-latex/unnamed-chunk-17-1}

\hypertarget{you-can-also-visualize-the-jacobian-determinant-image}{%
\subsection{You can also visualize the jacobian determinant
image}\label{you-can-also-visualize-the-jacobian-determinant-image}}

\hypertarget{jac-of-syn}{%
\subsubsection{jac of SyN}\label{jac-of-syn}}

\begin{Shaded}
\begin{Highlighting}[]
\KeywordTok{antsplot.single}\NormalTok{(syn_reg_jac)}
\end{Highlighting}
\end{Shaded}

\includegraphics{ANTsR_TUT_files/figure-latex/unnamed-chunk-18-1}

\hypertarget{jac-of-syncc}{%
\subsubsection{jac of SyNCC}\label{jac-of-syncc}}

\begin{Shaded}
\begin{Highlighting}[]
\KeywordTok{antsplot.single}\NormalTok{(synCC_reg_jac)}
\end{Highlighting}
\end{Shaded}

\includegraphics{ANTsR_TUT_files/figure-latex/unnamed-chunk-19-1}



\end{document}
